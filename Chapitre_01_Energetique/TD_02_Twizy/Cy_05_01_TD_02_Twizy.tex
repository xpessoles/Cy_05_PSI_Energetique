\documentclass[10pt,fleqn]{article} % Default font size and left-justified equations
\usepackage[%
    pdftitle={Energétique},
    pdfauthor={Xavier Pessoles}]{hyperref}

    
\input{style/new_style}
\input{style/macros_SII}
\usepackage{multicol}
\usepackage{siunitx}
%\usepackage{picins}
\fichetrue
%\fichefalse

\proftrue
%\proffalse

\tdtrue
%\tdfalse

\courstrue
\coursfalse


\def\classe{\textsf{PSI$\star$ -- MP}}
\def\xxnumpartie{Cycle 05}
\def\xxpartie{Modéliser le comportement des systèmes mécaniques dans le but d'établir une loi de comportement ou de déterminer des actions mécaniques en utilisant les méthodes énergétiques}

\def\xxnumchapitre{Chapitre 1 \vspace{.2cm}}
\def\xxchapitre{\hspace{.12cm} Approche énergétique}

\def\discipline{Sciences \\Industrielles de \\ l'Ingénieur}
\def\xxtete{Sciences Industrielles de l'Ingénieur}


  
\def\xxposongletx{2}
\def\xxposonglettext{1.45}
\def\xxposonglety{20}
%\def\xxonglet{Part. 1 -- Ch. 3}
\def\xxonglet{\textsf{Cycle 05}}

\def\xxactivite{TD}
\def\xxauteur{\textsl{Xavier Pessoles}}


\def\xxtitreexo{Renault Twizy}
\def\xxsourceexo{\hspace{.2cm} \footnotesize{Concours Mines Ponts -- PSI 2017}}


\def\xxcompetences{%
\vspace{-.5cm}
\footnotesize{
\textsl{%
\textbf{Savoirs et compétences :}\\
\vspace{-.2cm}
\begin{itemize}[label=\ding{112},font=\color{ocre}] 
\item Mod2.C18.SF1 : Déterminer l’énergie cinétique d’un solide, ou d’un ensemble de solides, dans son mouvement par rapport à un autre solide.
\item Res1.C1.SF1 : Proposer une démarche permettant la détermination de la loi de mouvement.
%\item Mod1.C5.SF2 : Déterminer la puissance des actions mécaniques extérieures à un solide ou à un ensemble de solides, dans son mouvement rapport à un autre solide.
%\item Mod1.C5.SF3 : Déterminer la puissance des actions mécaniques intérieures à un ensemble de solides.
\end{itemize}}}}

\def\xxfigures{
\includegraphics[width=.7\textwidth]{images/fig_01}
}%figues de la page de garde


\def\xxpied{%
Cycle 05 -- Modélisation mécanique -- Énergétique\\% afin de valider leurs performances.\\
Chapitre 1 -- \xxactivite%
}

\setcounter{secnumdepth}{5}
%---------------------------------------------------------------------------


\begin{document}
%\chapterimage{png/Fond_Cin}
\input{style/new_pagegarde}
\vspace{4.5cm}
\pagestyle{fancy}
\thispagestyle{plain}


\def\columnseprulecolor{\color{ocre}}
\setlength{\columnseprule}{0.4pt} 

%\ifprof
%\else
\begin{multicols}{2}
%\fi
\section*{Mise en situation}
\ifprof
\else

\begin{center}
%\includegraphics[width=\linewidth]{images/fig_02}
\end{center}

\fi

\subsection*{Choix du motoréducteur}

\subsubsection*{Validation du choix constructeur du moto-réducteur}
\begin{obj}
Mettre en place un modèle permettant de choisir un ensemble moto-réducteur afin d’obtenir les
exigences d’accélération et de vitesse.
\end{obj}

***L’Annexe 4 donne le paramétrage et les données nécessaires pour cette modélisation.


3.1 Choix de l’ensemble moto-réducteur
3.1.1 Equation de mouvement du véhicule
\begin{obj}
Objectif : Déterminer l’équation de mouvement nécessaire pour choisir l’ensemble moto-réducteur.
\end{obj}

\textbf{Notations :}
\begin{itemize}
\item puissance extérieure des actions mécaniques du solide $i$ sur le solide $j$ dans le mouvement de $i$ par rapport à 0 : $\mathcal{P}\left( i \to j / 0\right)$;
\item puissance intérieure des actions mécaniques entre le solide $i$ et le solide $j$: $\mathcal{P}\left( i \leftrightarrow j\right)$ ;
\item énergie cinétique du solide $i$ dans son mouvement par rapport à 0 : $\mathcal{E}_c\left(i/0\right)$.
\end{itemize}

\subparagraph{}\textit{Rédiger les réponses aux questions suivantes dans le cadre prévu à cet effet du document réponse :
\begin{itemize}
\item écrire la forme générale du théorème de l’énergie puissance appliqué au véhicule en identifiant les différentes puissances exterieures, les différentes puissances intérieures et les énergies cinétiques des différents éléments mobiles en respectant les notations précédentes ;
\item déterminer explicitement les différentes puissances extérieures ;
\item déterminer explicitement les différentes puissances intérieures ;
\item déterminer explicitement les énergies cinétiques ;
\item en déduire une équation faisant intervenir $C_m$, $N_1$, $N_2$, $v$, $\omega_m$, $\omega_{1/0}$, $\omega_{2/0}$ …… ;
\item expliquer pourquoi l’équation obtenue n’est pas l’équation de mouvement du véhicule.
\end{itemize}
}
\ifprof
\begin{corrige}~\\
\end{corrige}
\else
\fi


\subparagraph{}\textit{À partir des théorèmes généraux de la dynamique, déterminer une équation supplémentaire qui permet simplement de déterminer $(N_1 + N_2)$. Puis avec l’équation précédente, écrire l’équation de mouvement du véhicule.}
\ifprof
\begin{corrige}~\\
\end{corrige}
\else
\fi

\subparagraph{}\textit{Déterminer en énonçant les hypothèses nécessaires les relations entre $(v, \omega_{10})$, $(v, \omega_{20})$ et $(\omega_{m}, \omega_{10})$. Montrer que l’équation de mouvement du véhicule peut se mettre sous la forme $\dfrac{rC_m(t)}{R}-F_r(t)=M_{eq}\dfrac{\text{d} v(t)}{\text{d} t}$ avec $F_r(t)$ fonction de $m$, $\mu$, $g$, $R$ et $\alpha$ et $M_{eq}$ fonction $m$, $J_m$, $J_R$, $R$ et $r$.}
\ifprof
\begin{corrige}~\\
\end{corrige}
\else
\fi

\subsubsection*{Détermination du coefficient de résistance au roulement $\mu$}
\begin{obj}
Déterminer le coefficient de résistance au roulement $\mu$ suite à une expérimentation.
\end{obj}


\subparagraph{}\textit{En utilisant les résultats de l’essai effectué en ***Annexe 3, il est possible de déterminer le coefficient de
résistance au roulement $\mu$. Proposer un protocole expérimental pour l’évaluer :
\begin{itemize}
\item justifier dans quelle phase se placer;
\item définir la variable mesurée;
\item définir les hypothèses nécessaires;
\item énoncer les équations utilisées pour déterminer $\mu$.
\end{itemize}}
\ifprof
\begin{corrige}~\\
\end{corrige}
\else
\fi

\subsubsection*{Choix du moto-réducteur}
\begin{obj}
Choisir un ensemble moto-réducteur afin d’obtenir les exigences d’accélération et de vitesse.
\end{obj}

Les courbes de l’évolution de l’accélération maximale $\dfrac{\text{d} v(t)}{\text{d} t}$
du véhicule obtenue pour 3 moteurs présélectionnés en fonction du rapport de transmission $r$ issues de l’équation de mouvement du véhicule précédente sont fournies sur le document réponse.




\subparagraph{}\textit{Déterminer la valeur minimale du rapport de transmission $r_{\text{mini}}$ pour les 3 moteurs proposés qui permet d’obtenir l’accélération maximale moyenne souhaitée dans les exigences de ***l’Annexe 1.}
\ifprof
\begin{corrige}~\\
\end{corrige}
\else
\fi


\subparagraph{}\textit{En fonction des données de l’Annexe 4***, déterminer la valeur maximale du rapport de transmission $r_{\text{max}}$ qui permet d’obtenir au moins la vitesse maximale du véhicule souhaitée dans les exigences de l’Annexe 1***.}
\ifprof
\begin{corrige}~\\
\end{corrige}
\else
\fi


\subparagraph{}\textit{À partir des résultats précédents, choisir parmi les 3 moteurs proposés, celui qui respecte les exigences d’accélération et de vitesse souhaitées permettant la plus grande plage possible pour le rapport de transmission.}
\ifprof
\begin{corrige}~\\
\end{corrige}
\else
\fi


\subsubsection*{Validation du choix constructeur du moto-réducteur}
\begin{obj}
Valider le choix du moto-réducteur fait par le constructeur.
\end{obj}

\subparagraph{}\textit{À partir de la vue 3D du réducteur choisi par le constructeur, fourni en *** Annexe 5, compléter le schéma cinématique du document réponse, calculer son rapport de transmission $r = \dfrac{\omega_{4/3}}{\omega_{4/3}}$ et conclure.}
\ifprof
\begin{corrige}~\\
\end{corrige}
\else
\fi



\end{multicols}
\end{document}

\subparagraph{}\textit{}
\ifprof
\begin{corrige}~\\
\end{corrige}
\else
\fi

\begin{center}
\includegraphics[width=\linewidth]{images/img_04}
%\textit{}
\end{center}


\subparagraph{}\textit{}
\ifprof
\begin{corrige}~\\
\end{corrige}
\else
\fi

\subparagraph{}\textit{}
\ifprof
\begin{corrige}~\\
\end{corrige}
\else
\fi

\subparagraph{}\textit{}
\ifprof
\begin{corrige}~\\
\end{corrige}
\else
\fi

\subparagraph{}\textit{}
\ifprof
\begin{corrige}~\\
\end{corrige}
\else
\fi

\subparagraph{}\textit{}
\ifprof
\begin{corrige}~\\
\end{corrige}
\else
\fi

\subparagraph{}\textit{}
\ifprof
\begin{corrige}~\\
\end{corrige}
\else
\fi
\begin{center}
%\includegraphics[width=\linewidth]{images/fig_05}
\end{center}
